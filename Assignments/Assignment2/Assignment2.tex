\documentclass[11pt, oneside]{article}   	% use "amsart" instead of "article" for AMSLaTeX format
\usepackage{geometry}                		% See geometry.pdf to learn the layout options. There are lots.
\geometry{letterpaper, margin=1.0in}                   		% ... or a4paper or a5paper or ... 
%\geometry{landscape}                		% Activate for for rotated page geometry
%\usepackage[parfill]{parskip}    		% Activate to begin paragraphs with an empty line rather than an indent
\usepackage{graphicx}				% Use pdf, png, jpg, or eps§ with pdflatex; use eps in DVI mode
								% TeX will automatically convert eps --> pdf in pdflatex		
\usepackage{amssymb, amsmath}

\newcommand{\vectr}[1]{{\mathbf{#1}}}


\usepackage{fancyhdr}
\pagestyle{fancy}
\chead{CESG~506}
\lhead{Assignment \#2}
\rhead{Spring 2020}


\begin{document}
\noindent
With the last assignment, you  explored two simple nonlinear systems.  You solved Problem~1-1 using displacement control (in its most primitive), and Problem~1-2 using incremental load control (prescribing the load level).

This week, you'll be exploring path following techniques for tracking equilibrium paths beyond a snap-through\footnote{This is where load control will fail.} or a snap-back\footnote{This is where displacement control will fail.} point.

\section*{Problem 2-1: Displacement control for a two-degrees-of-freedom (2 DOF) problem}

\begin{center}
\includegraphics[height=1.00in]{Problem2-1.png}
\end{center}
\noindent
This problem expands Problem~1-2 from Assignment~\#1 such that you shall track the entire equilibrium path from 
$(u=0, v=0)$ and $\lambda=0$ at least through $\lambda\ge 2.0$.

\begin{subequations}
\begin{enumerate}
\item 
    Using Henky strain, a linear relation $\sigma = E \varepsilon$, $A=const.$, and equilibrium on the deformed system, derive the relationship between displacements and forces on the free node.   Adjust it for path following as
 \begin{equation}
	\vectr{R}(\gamma,\vectr{u}) = \gamma \bar {\vectr{P}} - \vectr{F}(\vectr{u}(s)) = \vectr{0}
\end{equation}
with reference load, 
$\bar{\vectr{P}}=-(0.99$~kN$)\,\vectr{j}$,  load intensity factor $\gamma$ and 
displacement $\vectr{u}$.

Find the tangents $\partial\vectr{R}/\partial\gamma$ and $\partial\vectr{R}/\partial\vectr{u}$ as you will need them in what follows.   

You should have all the necessary parts for this question from Assignment~\#1.

\item
Develop a, or adjust your existing Newton method to incrementally find $\gamma$ and $\vectr{u}$ 
using displacement control on the vertical displacement, $v=\bar{v}$.  The respective constraint equation is
\begin{equation}
   g(\vectr{u}) := \vectr{e}_v\cdot\vectr{u} - \bar{v} = 0 
\end{equation}
with 
\begin{equation}
   \Delta\vectr{u} \Rightarrow \left\{ \begin{array}{c} u \\ v \end{array} \right\}
   \quad\text{and}\quad
   \Delta\vectr{e}_v \Rightarrow \left\{ \begin{array}{cc} 0 & 1  \end{array} \right\}
   ~.
\end{equation}

\item
    Plot load level $\gamma$ versus vertical displacement $v$, as well as $\gamma$ versus horizontal displacement $u$ for the converged points on the equilibrium path.  Observe how only one of them actually looks like the curve you (might have) expected. 

\item
    Add a plot $u$ versus $v$ to explore yet another view of the solution path.  Try and overlay this curve on a contour
    plot  of $F_x(u,v)$ and on a contour plot for $F_y(u,v)$\footnote{The first should show that the solution path is hugging the $P_x(u,v)=0$ contour line.  The latter is like a road going through a hilly landscape with elevation along the road being $\lambda(s) \bar P$.}.
    

\section*{Problem 2-2: Going into higher dimensions }

Now let's take the problem into the third dimension and look at a 3D-truss system with 6 nodes and 7 truss members as shown.
\begin{center}
	\includegraphics[height=3.00in]{Problem2-2.png}
\end{center}

Only node 5 shall be loaded by a vertical load, $P_5=\gamma \bar P$.
\smallskip

\begin{enumerate}
\item
Adjust your code to accommodate the higher dimensional system. 
At this point, it is highly advisable to drop the brute-force approach and use appropriate functions or my TrussElement class to represent each system component.  Assembly may still be done beforehand, though transitioning to a more generic concept will be helpful for future assignments.

\item
Use the vertical displacement $v_5=\bar{v}$ as control parameter and find the equilibrium path using displacement control.  Trace the equilibrium path until members 3--6 and 4--6 are in tension.

\item
Present  load-displacement diagrams by plotting the load factor $\gamma$ against displacement components of nodes~5 and 6.

\item
Plot the planar view of the equilibrium path for nodes 5 and 6 (can be in one or two plots).

\end{enumerate}

    
\end{enumerate}\end{subequations}





\end{document}  